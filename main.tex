\documentclass[a4paper]{article}

\usepackage[english]{babel}
\usepackage[utf8]{inputenc}
\usepackage{amsmath}
\usepackage{graphicx}
\usepackage[colorinlistoftodos]{todonotes}

\title{What makes blockchain different from current technologies}

\author{Mohammad H. Tabatabaei, Narasimha Raghavan, Roman Vitenberg}

\date{University of Oslo}

\begin{document}
\maketitle

\begin{abstract}
A short summary comes here. What topic we want to investigate and why? What is our main results and conclusion?
\end{abstract}

\section{Introduction}
\label{sec:introduction}

Explain the context.

\section{What is blockchain?}
\label{sec:wib}

A chain of blocks that store transactions was first introduced in bitcoin, and it is known as blockchain. A blockchain provides the ability of participating in sending transactions or verifying them among the peers without the help of a third party in a decentralized system. Participants of the blockchain network do not need to trust each other and at the same time they are guaranteed that their stored data in the blockchain is immutable and all of the peers have the same view of the blockchain's state. 

\subsection{Bitcoin and Blockchain are NOT the Same}
\label{sec:bitcoinandblockchain}

Differences between bitcoin and blockchain will be defined here in terms of:
\begin{enumerate}
\item Use case and context
\item Identity of the users
\item Consensus protocol 
\end{enumerate}

\subsection{Public VS. Private blockchain}
\label{sec:publicvsprivate}

Blockchain technologies contain tamper proof digital ledgers that continuously grow in terms of the number of records, that are linked together and secured using cryptography. These characteristics in turn enabled blockchain technologies to  gain much attention in several applications domains. Furthermore, many literature classify the blockchain technologies into two types: Public (Permission less) and Private (Permissioned). 

In a public blockchain network,  any node can join and leave the network with full permissions to read and write the content at any time in the network. Furthermore, each transaction is verified and synced with all other nodes before the transaction is written to the network. In other words, there is no centralized group membership coordinator exists to manage the membership process. Furthermore, every transaction needs to be verified and synced with all other nodes in the network. These characteristics in turn introduce the following advantages and disadvantages. 

Advantages of a public blockchain include that the integrity of the content in the network is not affected since all nodes  need to participate in verifying the content before the content is added to the network. This in turn makes sure that the retrieved data is correct. Additionally, all participants of the network get access to the all piece of information  available in the network, which in turn ensures transparency and consistency. 

Disadvantages of a public blockchain include that every node added to the network increase the energy and storage consumption. Furthermore, the transparent characteristics will not suit the enterprise use cases, where in the not all participants of the network get to access all piece of information in the network. Additionally, it is relatively slow to create new content in the blockchain network. 

The other terminologies used to refer to public blockchains are permissionless and authenticationless.  Examples of public blockchain includes Bitcoin, Ethereum and Zcash. 

In a private blockchain network, a trusted party control the membership of who can join and leave  the network in addition to the type of access (read or write) to the type of content in the network. 
Furthermore, every transaction will involve only the nodes that are impacted by the transaction. 

Private blockchains are mainly helpful for scenarios where the blockchain network needs to be managed by a central entity, which could be either a private or a public organization. As part of the management, private blockchains enable the following capabilities to the network:
\begin{itemize}
\item All entities (a natural person or an organization)  should have an identity and should be authenticated.
\item Access controls for entities differ based on the role played by the entities in the network.
\item Transactions are endorsed only by the relevant entities, which in turn saves time to complete the transaction.
\end{itemize}

The alternative terminology for a private blockchain is permissioned blockchain. Hyperledger fabric is the most common sample of a permissioned blockchain that can be used for implementing blockchain applications in different contexts without needing it to be public for no reason. 

\subsection{Participants and their Roles}
\label{sec:participants}
A blockchain is a decentralized system which contains untrusted peers with trusted transactions among them. in order to solve the paradox of enabling trusted and secure transactions among untrusted participants, the blockchain technology has its own fundamental characteristics which have to be maintained by its participants. In every blockchain system, both public and private, there should be the following participants with their special roles:

\begin{description}
\item[Creators of Transactions] They are the ones whom the whole blockchain system has been created for and use the system to share their data or proceeding an action in the network. They just use the network to create a transaction and broadcast it to the whole system.
\item[Proposers] They have the responsibility of gathering and validating the broadcasted transactions. Then, they generate a block with the validated transactions. This block have to be in an appropriate size, which is characterized by the blockchain. After the creation of the block, the proposer needs to be selected among the other competitors with a special mechanism which is also defined by the blockchain algorithm to get the right to propose its generated block. After winning the competition, the proposer can broadcast the valid block.   
\item[Acceptors] They receive the valid broadcasted block and try to verify it. The acceptors need to reach consensus on the valid block to verify and accept it in order to be appended to the blockchain.
\item[Recorders of Blocks] These entities record the block and append it to the blockchain after receiving and verifying them. They also store the whole blockchain in their machine. 
\item[Query Issuer] They are able to issue query of all different types over the existing blockchain network. 
\end{description} 

\newpage

\begin{thebibliography}{9}
\bibitem{nano3}
  
\end{thebibliography}
\end{document}
